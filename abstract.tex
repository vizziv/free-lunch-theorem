The \emph{parallel chip-firing game} is an automaton on graphs in which
vertices ``fire'' chips to their neighbors. This simple model contains much
emergent complexity and has many connections to different areas of
mathematics. In this work, we study \emph{firing patterns}, which
describe each vertex's interaction with its neighbors in this game. First, we
introduce the concepts of \emph{motors} and \emph{motorized games}. Motors
generalize the game and allow us to isolate local behavior of the
(ordinary) game. We study the effects of motors connected to a tree, and show
that motorized games can be transformed into ordinary games if each motor's
firing pattern occurs in some ordinary game. Then, we completely characterize
the firing patterns that can occur in an ordinary game, which have a
surprisingly simple combinatorial description.
