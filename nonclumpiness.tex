\section{Signed Sums of Binary Sequences}\label{binSeq}
Throughout this section, if $b \in \{0,1\}$, we may write $\ol{b} = 1-b$.

We take a break from the parallel chip-firing game to consider binary
strings. We denote the $i$\xth element of a binary string $p$ as $p_i$. For
simplicity, any integer equivalent to $i \bmod n$ may replace $i$. Let a
\emph{$b$-sector} of $p$, where $b \in \{0,1\}$, be an integer interval $[x,y]$
such that
\begin{align*}
  p_x = p_{y-1} = p_y &= b \\
  p_{x-1} = p_{x-2} &= \ol{b} \\
  \forall i \in [x+1, y-3]: p_i + p_{i+1} &\neq 2\ol{b}.
\end{align*}
That is, the image of a 0-sector is preceded by two 1s, starts with 0, ends
with two 0s and contains no two consecutive 1s. The same statements with 0s and
1s swapped are true for 1-sectors. Roughly speaking, $b$-sectors have no two
$\ol{b}$s in a row and extend as far back as possible.

It is easy to see that almost any string can be partitioned into 0- and
1-sectors in exactly one way, with exceptions only for always-alternating
strings (e.g.\ 010101) that can be thought of as one 0-sector or one
1-sector. Figure~\ref{sectorEx} has an example.

\begin{figure}
  \[
    \underbracket[1pt]{01000100}\overbracket[1pt]{11011}\underbracket[1pt]{00}
    \overbracket[1pt]{101011}
  \]
  \caption{A string's 0-sectors (marked below) and 1-sectors (marked
    above).}
  \label{sectorEx}
\end{figure}

Let
\begin{align*}
  s_i(p) &= \begin{cases}
    -1 & \text{if $i$ is in a 0-sector of $p$} \\
    1 & \text{if $i$ is in a 1-sector of $p$}
  \end{cases} \\
  \delta_i(p) &= \begin{cases}
    0 & \text{if $i$ is in a $b$-sector of $p$ and $i+1$ is in a $b$-sector of
      $p$} \\
    1 & \text{if $i$ is in a $b$-sector of $p$ and $i+1$ is in a
      $\ol{b}$-sector of $p$}.
  \end{cases}
\end{align*}
Our main theorem in this section concerns the sum
\begin{equation}\label{misbsum}
  M_S(p,q) = \sum_{i \in S}(s_i(p)(p_i - q_{i-1}) + s_i(q)(q_i - p_{i-1}) -
  \delta_i(p) - \delta_i(q)),
\end{equation}
where $p,q$ are length-$n$ binary sequences and $S \subseteq [0, n-1]$. This
sum, superficially speaking, measures each sequence's ``disagreement'' with the
other shifted back one step minus the number of sector switches. The rules of
the parallel chip-firing game put a global upper bound on the total
disagreement between vertices, yet the following theorem says that sector
switches require disagreement. We show in Section~\ref{nonclumpiness} that this
implies that firing sequences with sector switches are impossible once a game
has become periodic.

\begin{thm}\label{morale}
$M_{[0,n-1]}(p,q) \geq 0$.
\end{thm}

\begin{proof}
We can caculate $M_{[0,n-1]}(p,q)$ as $\sum_{i=0}^{n-1}M_{\{i\}}(p,q)$, and
$M_{\{i\}}(p,q)$ is determined by $p_{i-1}$, $p_i$, $s_i(p)$, $s_{i+1}(p)$, and
the same data for $q$. The motivation for using $s_{i+1}(p)$ as opposed to
$s_{i-1}(p)$ in $\delta_i(p)$ is that a switch away from a $b$-sector can occur
between $i$ and $i+1$ only if $p_{i-1} = p_i = b$. Let
\[
  \mu_i(p,q) = (p_{i-1},p_i,s_i(p),s_{i+1}(p),q_{i-1},q_i,s_i(q),s_{i+1}(q))
\]
and let $\G$ be a weighted digraph with
\begin{align*}
  \V[\G] &= \set{\mu_i(p,q)}{p,q \text{ strings}, i \in \nats} \\
  \E[\G] &= \set{(u,v,w)}{\exists p,q,i\colon
    u = \mu_i(p,q),
    v = \mu_{i+1}(p,q),
    w = M_{\{i\}}(p,q)}.
\end{align*}
(The third item of each edge is its weight.) Note that not every possible tuple
is a vertex. Define the weight of a path to be the sum of the weights of its
member edges, and call a path negative if it has negative weight. We can
calculate the $M_{[0,n-1]}(p,q)$ as the weight of a path induced by the
sequence of vertices $(\mu_0(p,q), \dots, \mu_{n-1}(p,q),
\mu_0(p,q))$. Therefore, it suffices to show that $\G$ has no negative
cycles. Running the Bellman-Ford algorithm~\cite{bellmanford} on $\G$ shows
this to be the case. We describe $\G$ and the algorithm in a Python program in
Appendix~\ref{bfAlg}.
\end{proof}

\section{Nonclumpiness of Periodic Firing Patterns}\label{nonclumpiness}
We consider parallel chip-firing game $\s$ on undirected graph $G$. The
\emph{periodic firing pattern} (PFP) of a vertex $v \in \V$ is the binary
string
\[
  \firing{v}{t_0}\dots\firing{v}{t_0 + \period - 1},
\]
where $t_0$ is the smallest natural number such that $\pos{t_0}$ is
periodic\footnote{The reason we introduce PFPs instead of continuing to reason
  with firing sequences is because a PFP is aware of the period of the game it
  occurs in. For instance, the PFPs 01 and 0101 result in the same periodic
  firing sequence, but while the latter, which has period 4, might occur in the
  same game as the PFP 0011, the former, which has period 2, cannot.}. We write
the PFP of $v$ as $\pfp{v}$. For simplicity, we assume here that $t_0 = 0$ and
index PFPs modulo $\period$.

Let $\pfps$ be the set of all PFPs occurring in $\s$. Call a PFP with both
consecutive 0s and consecutive 1s \emph{clumpy}, and let $\cpfps$ be the set of
all clumpy PFPs occurring in $\s$. (Recall that the $\period$\xth and 0\xth
entries of a PFP are the same, so, for example, 011010 is clumpy.) It is known
that, given almost any\footnote{The given construction requires that the PFP
  not be decomposable to a repeated substring. Using Theorem~\ref{natMotors},
  one can expand the construction to any nonclumpy PFP other than those that
  are 01 or 10 repeated more than once. These PFPs turn out to be impossible,
  though the corresponding firing sequences are possible in games of period 2.}
nonclumpy PFP, one can construct a parallel chip-firing game on a simple cycle
in which every vertex has that PFP shifted by some number of
steps~\cite{cycle}. We prove here that clumpy PFPs cannot occur in any parallel
chip-firing game.

\begin{lem}\label{strongbg}
For all $v \in \V$ and $a, b \in \nats$,
\begin{equation}\label{local}
  -\deg{v} + 1 \leq \sum_{t=a}^{b}(\receiving{v}{t-1} - \deg{v}\firing{v}{t})
  \leq \deg{v}-1.
\end{equation}
\end{lem}

\begin{proof}
We express $\chips{v}{b}$ in terms of $\chips{v}{a-1}$.
\begin{align*}
  \chips{v}{b} &= \chips{v}{a-1} +
  \smashoperator{\sum_{t=a-1}}^{b-1}(\receiving{v}{t} - \deg{v}\firing{v}{t})
  \\
  \chips{v}{b} - \deg{v}\firing{v}{b} &= \chips{v}{a-1} -
  \deg{v}\firing{v}{a-1} + \sum_{t=a}^{b}(\receiving{v}{t-1} -
  \deg{v}\firing{v}{t})
\end{align*}
Recall that $0 \leq \chips{v}{t} - \deg{v}\firing{v}{t} \leq \deg{v} - 1$ for
all $t \in \nats$ such that $\pos{t}$ is periodic, which gives the desired
inequality.
\end{proof}

We define
\begin{align*}
\pfpverts{p} &= {\set{v \in \V}{\pfp{v}=p}} \\
\pfpedges{p}{q} &= {\set{\{v,w\} \in \E}{\pfp{v}=p, \pfp{w} = q}} \\
\misb{S}{p}{q} &= \sum_{i \in S}(p_{i} - q_{i-1})
\end{align*}
for binary strings $p$ and $q$. We now prove our main result: clumpy PFPs do
not occur in the parallel chip-firing game.

\begin{thm}\label{nct}
$\size{\pfpverts{p}} = 0$ for all $p \in \cpfps$.
\end{thm}

\begin{proof}
Roughly, summing an inequality given by Lemma~\ref{strongbg} over all vertices
with clumpy PFPs bounds a quantity below, and summing an inequality given by
the Theorem~\ref{morale} over all edges incident with a vertex with a clumpy
PFP gives an upper bound on the same quantity. The lower bound is the total
number of vertices with clumpy PFPs, and the upper bound is 0.

Let $a,b \in \nats$ and $v \in \V$. Grouping the sum in \eqref{local} by $v$'s
neighbors instead of time steps yields
\[
  -\deg{v} + 1 \leq -\smashoperator{\sum_{w \in
      \N{v}}}\misb{[a,b]}{\pfp{v}}{\pfp{w}} \leq \deg{v} - 1.
\]
Regrouping gives us
\[
  1 \leq \smashoperator{\sum_{w \in \N{v}}}(1 +
  r\misb{[a,b]}{\pfp{v}}{\pfp{w}}),
\]
where $r = \pm1$. Let $p \in \pfps$. The above summed over $v \in \pfpverts{p}$
is
\[
  \size{\pfpverts{p}} \leq \smashoperator{\sum_{\substack{v \in \pfpverts{p}
        \\\ w \in \N{v}}}}(1 + r_v\misb{[a,b]}{p}{\pfp{w}}),
\]
where each $r_v = \pm1$ can depend on $v$. (Notation: ranges for outer sums are
above ranges for inner sums.)

For all $p \in \pfps$, let $\sctr{p}$ be the set of sectors of $p$. Abusing
notation slightly, we may write $s_X(p)$ instead of $s_i(p)$ if $i \in X \in
\sctr{p}$. Because each $X \in \sctr{p}$ is of the form $[a,b]$ for some $a,b
\in \nats$, we can sum the above inequality over $X \in \sctr{p}$ and $p \in
\cpfps$ to get
\begin{equation}\label{almost}
  \smashoperator[r]{\sum_{p \in \cpfps}}\size{\pfpverts{p}} \leq
  \smashoperator{\sum_{\substack{
        p \in \cpfps \\ v \in \pfpverts{p} \\
        w \in \N{v} \\ X \in \sctr{p}
  }}}(1 + r_{v,X}\misb{X}{p}{\pfp{w}}),
\end{equation}
where each $r_{v,X} = \pm1$ can depend on $v$ and $X$.

Let $p \in \cpfps$ and $q \in \pfps$. If $q$ is clumpy, then
\[
  M_{[0,\period-1]}(p,q) =
  \smashoperator{\sum_{X \in \sctr{p}}}(s_X(p)\misb{X}{p}{q} - 1) +
  \smashoperator{\sum_{X \in \sctr{q}}}(s_X(q)\misb{X}{q}{p} - 1).
\]
The $-1$ in each sum accounts for the $-\delta_i(p)-\delta_i(q)$ term in
\eqref{morale}, the definition of $M$. If instead $q$ is not clumpy, then
$\sctr{q} = \{[0,\period-1]\}$, so
\[
  M_{[0,\period-1]}(p,q) =
  \smashoperator{\sum_{X \in \sctr{p}}}(s_X(p)\misb{X}{p}{q} - 1) +
  s_{[0,\period-1]}(q)\misb{[0,\period-1]}{q}{p}.
\]
However, $\misb{[0,\period-1]}{q}{p} = 0$ because $p$ and $q$ have the same
length and number of 1s.

Let $W = \set{v \in \V}{\pfp{v} \in \cpfps}$ be the set of vertices with clumpy
PFPs. Choosing $r_{v,X} = -s_X(\pfp{v})$ and splitting the sum in
\eqref{almost} between neighbors with and without clumpy PFPs yields
\begin{align*}
  \smashoperator[r]{\sum_{\substack{
        p \in \cpfps \\
        X \in \sctr{p}
  }}}\size{\pfpverts{p}} &\leq
  \smashoperator{\sum_{\substack{
        p \in \cpfps \\ v \in \pfpverts{p} \\
        w \in \N{v} \cap W \\ X \in \sctr{p}
  }}}(1-s_X(p)\misb{X}{p}{\pfp{w}}) +
  \smashoperator{\sum_{\substack{
        p \in \cpfps \\ v \in \pfpverts{p} \\
        w \in \N{v} \setminus W \\ X \in \sctr{p}
  }}}(1-s_X(p)\misb{X}{p}{\pfp{w}}) \\
  &=
  \smashoperator[l]{\sum_{\substack{
        p,q \in \cpfps \\
        e \in \pfpedges{p}{q}
  }}}\!\!\Bigg(\smashoperator[r]{\sum_{X \in \sctr{p}}}(1-s_X(p)\misb{X}{p}{q})
  + \smashoperator{\sum_{X \in \sctr{q}}}(1-s_X(p)\misb{X}{q}{p})\Bigg)
  \\ &\qquad +
  \smashoperator{\sum_{\substack{
        p \in \cpfps \\ v \in \pfpverts{p} \\
        w \in \N{v} \setminus W \\ X \in \sctr{p}
  }}}(1-s_X(p)\misb{X}{p}{\pfp{w}}) \\
  &= -\smashoperator{\sum_{\substack{
        p,q \in \cpfps \\
        e \in \pfpedges{p}{q}
  }}}M_{[0,\period-1]}(p,q) -
  \smashoperator{\sum_{\substack{
        p \in \cpfps \\
        v \in \pfpverts{p} \\ w \in \N{v} \setminus
        W}}}M_{[0,\period-1]}(p,F(w)) \\
  &\leq 0.
\end{align*}
The last line follows from Theorem~\ref{morale}, and when we sum over $p,q \in
\cpfps$, we consider $p$ and $q$ unordered. (A sum over $\{p,q\} \subseteq
\cpfps$ is one alternative notation.) Sets have nonnegative sizes, so
$\size{\pfpverts{p}} = 0$ for all $p \in \cpfps$.
\end{proof}
