\section{Signed Sums of Binary Sequences}\label{binSeq}
We take a brief break from the parallel chip-firing game itself to consider
binary strings.  Throughout this section, $p$ and $q$ are length-$n$ binary
strings, $b \in \{0,1\}$, and $\ol{b} = 1-b$. We denote the $i$\xth element of
a binary string $p$ as $p_i$, and any integer equivalent to $i \bmod n$ may
replace $i$.

The following definition formalizes the notion of part of a string being
``mostly'' 0s or 1s. A \emph{$b$-sector} of $p$ is a $\subseteq$-maximal
integer interval $[x,y]$ such that $p_{y-1} = p_y = b$ and either $p_i = b$ or
$p_{i+1} = b$ for all $i \in [x,y]$. Informally, $b$-sectors end with
consecutive $b$s and extend back as far as possible. We have to make an
exception for always-alternating strings, such as 01010101, that have neither 0
nor 1 twice in a row. We arbitrarily define $[0, n-1]$ to be a 0-sector of
them. It is the transitions between 0- and 1-sectors that are ultimately
important, so this exception is acceptable. It is simple to verify that the
indices of every binary string have a unique decomposition into 0- and
1-sectors. An example is shown in Figure~\ref{sectorEx}.

\begin{figure}[tbh]
  \[
    \aunderbrace[l1r]{01000100}
    \!\aoverbrace[L1R]{11011}
    \!\aunderbrace[l1r]{00}
    \!\aoverbrace[L1R]{101011}
  \]
  \caption{A string's 0-sectors (marked below) and 1-sectors (marked
    above).}
  \label{sectorEx}
\end{figure}

Let
\begin{align*}
  s_i(p) &= \begin{cases}
    -1 & \text{if $i$ is in a 0-sector of $p$} \\
    1 & \text{if $i$ is in a 1-sector of $p$}
  \end{cases} \\
  \delta_i(p) &= \begin{cases}
    0 & \text{if $i$ is in a $b$-sector of $p$ and $i+1$ is in a $b$-sector of
      $p$} \\
    1 & \text{if $i$ is in a $b$-sector of $p$ and $i+1$ is in a
      $\ol{b}$-sector of $p$}
  \end{cases} \\
  M_i(p,q) &= s_i(p)(p_i - q_{i-1}) + s_i(q)(q_i - p_{i-1}) - \delta_i(p) -
  \delta_i(q).
\end{align*}
Our main theorem in this section concerns what we refer to as the
\emph{mischief between $p$ and $q$},
\[
  M(p,q) = \sum_{i=0}^{n-1} M_i(p,q),
\]
Mischief, superficially speaking, measures three things: how much $p$ differs
from $q$ delayed one time step, which we call the \emph{misbehavior of $q$
  towards $p$}; vice versa; and the total number of sector switches. An example
calculation is shown in Figure~\ref{moraleEx}. The rules of the parallel
chip-firing game put a global upper bound on the total disagreement between
vertices, yet the following theorem states that mischief is nonnegative,
meaning that sector switches require disagreement. We show in
Section~\ref{nonclumpiness} that this implies that firing sequences with sector
switches are impossible once a game has become periodic.

% Dash
\newbracespec{d}{!{0.05em}@{\hspace{0.1em}}!{0.05em}}
% End
\newbracespec{e}{!{0.05em}}
% N dashes: e*{N}{d}e (sometimes with 1 instead of e)
\begin{figure}[tbh]
  \begin{align*}
    p\colon &\aunderbrace[e*{12}{d}1r]{010001}_4
    \!\aoverbrace[L1*{8}{d}1R]{00110}^2
    \!\aunderbrace[l1@{\hspace{0.1em}}1r]{11}_2
    \!\aoverbrace[L1*{10}{d}1R]{001010}^2
    \!\aunderbrace[l1*{2}{d}e]{11}
    & p\colon &\aunderbrace[l1r]{01000100}_2
    \!\aoverbrace[L1R]{11011}^4
    \!\aunderbrace[l1r]{00}_0
    \!\aoverbrace[L1R]{101011}^4 \\
    q\colon &\aunderbrace[1r]{1000100}_2
    \!\aoverbrace[L1R]{11011}^4
    \!\aunderbrace[l1r]{00}_0
    \!\aoverbrace[L1R]{101011}^4
    \!\aunderbrace[l1]{0}
    & q\colon &\aunderbrace[e*{14}{d}1r]{1000100}_2
    \!\aoverbrace[L1*{8}{d}1R]{11011}^4
    \!\aunderbrace[l1@{\hspace{0.1em}}1r]{00}_0
    \!\aoverbrace[L1*{10}{d}1R]{101011}^4
    \!\aunderbrace[l1]{0}
  \end{align*}
  \caption{We calculate the mischief between two strings by dividing each into
    sectors. The misbehavior of $p$ towards $q$ is $- (2-4) + (4-2) - (0 - 2) +
    (4 - 2) = 8$, as shown on the left. On the right, we see that the
    misbehavior of $q$ towards $p$ is 0. Each string has 4 sector switches, so
    $M(p,q) = 8 + 0 - 4 - 4 = 0$.}
  \label{moraleEx}
\end{figure}

\begin{thm}\label{morale}
The mischief between any two binary strings of equal length is nonnegative.
\end{thm}

\begin{proof}
Let
\[
  \mu_i(p,q) = (p_{i-1},p_i,s_i(p),s_{i+1}(p),q_{i-1},q_i,s_i(q),s_{i+1}(q)),
\]
a tuple of all information required to calculate $M_i(p,q)$, and let $\G$ be a
weighted digraph with
\begin{align*}
  \V[\G] &= \set{\mu_i(p,q)}{p,q \text{ strings}, i \in \nats} \\
  \E[\G] &= \set{(\mu_i(p,q),\mu_{i+1}(p,q),M_i(p,q))}{p,q \text{ strings}, i
    \in \nats}
\end{align*}
(The third item of each edge is its weight.) Define the weight of a path to be
the sum of the weights of its member edges, and call a path negative if it has
negative weight. We can calculate the mischief $M(p,q)$ as the weight of a path
in $\G$ induced by the sequence of vertices $(\mu_0(p,q), \dots,
\mu_{n-1}(p,q), \mu_0(p,q))$. Therefore, it suffices to show that $\G$ has no
negative cycles---in particular, no negative cycles of length $n$.

We discuss below the constraints that define what tuples can exist as some
$\mu_i(p,q)$. These constraints yield a graph $\G$ with 64 vertices and 256
edges, which is impractical to analyze by hand. However, running the
Bellman-Ford algorithm~\cite{bellmanford} on $\G$ shows it to be free of
negative cycles. A Python program specifies both the graph and the algorithm in
detail in the appendix.
\end{proof}

The only constraint on edges of $\G$ is that the state shared by the connected
vertices be consistent. For example, both $\mu_i(p,q)$ and $\mu_{i+1}(p,q)$
contain $p_i$. There are two constraints on the vertices of $\G$ enforced by
the definition of sectors in $p$. Analogous constraints apply for $q$.
\begin{itemize}
\item If $p_{i-1} = p_i = b$, then $s_i(p)$ indicates that $i$ is in a
  $b$-sector of $p$.
\item If $s_i(p) \neq s_{i+1}(p)$, then $p_{i-1} = p_i = b$ such that $s_i(p)$
  indicates that $i$ is in a $b$-sector of $p$.
\end{itemize}
To get a sense for how these force nonnegative mischief, consider the latter
constraint. The sector switch reduces mischief, but it also makes $p_{i+1} =
\ol{b}$. If $q_i = b$, then $q$ misbehaves towards $p$ at $i+1$. If $q_i =
\ol{b}$ and $i$ is in a $\ol{b}$-sector of $q$, then $p$ misbehaves towards $q$
at $i$. However, we cannot finish the last case, $q_i = \ol{b}$ with $i$ in a
$b$-sector of $q$, with similar local reasoning. The former constraint limits
the frequency with which that last case can happen, but there are many
scenarios to consider before dispatching the case. Using the Bellman-Ford
algorithm avoids this further casework.

\section{Nonclumpiness of Periodic Firing Patterns}\label{nonclumpiness}
We consider parallel chip-firing game $\s$ on undirected graph $G$. The
\emph{periodic firing pattern} (PFP) of a vertex $v \in \V$ is the binary
string
\[
  \firing{v}{t_0}\dots\firing{v}{t_0 + \period - 1},
\]
where $t_0$ is the smallest natural number such that $\pos{t_0}$ is
periodic\footnote{The reason we introduce PFPs instead of continuing to reason
  with firing sequences is because a PFP is aware of the period of the game it
  occurs in. For instance, the PFPs 01 and 0101 result in the same periodic
  firing sequence, but while the latter, which has period 4, might occur in the
  same game as the PFP 0011, the former, which has period 2, cannot.}. We write
the PFP of $v$ as $\pfp{v}$. For simplicity, we assume here that $t_0 = 0$ and
index PFPs modulo $\period$.

Let $\pfps$ be the set of all PFPs that are compatible with $\s$, which means
they have the same number of 0s and 1s as a PFP that occurs in $\s$. Call a PFP
with both consecutive 0s and consecutive 1s \emph{clumpy}, and let $\cpfps$ be
the set of clumpy PFPs that are compatible with $\s$. (Recall that the
$\period$\xth and 0\xth entries of a PFP are the same, so, for example, 011010
is clumpy.) It is known that, given almost any\footnote{The given construction
  requires that the PFP not be decomposable to a repeated substring. Using
  Theorem~\ref{natMotors}, one can expand the construction to any nonclumpy PFP
  other than those that are 01 or 10 repeated more than once. These PFPs turn
  out to be impossible, though the corresponding firing sequences are possible
  in games of period 2.}  nonclumpy PFP, one can construct a parallel
chip-firing game on a simple cycle in which every vertex has that PFP shifted
by some number of steps~\cite{cycle}. We prove here that clumpy PFPs cannot
occur in any parallel chip-firing game.

\begin{lem}\label{strongbg}
In a periodic game on $G$, for all $v \in \V$ and $a, b \in \nats$,
\begin{equation}\label{local}
  -\deg{v} + 1 \leq \sum_{t=a}^{b}(\receiving{v}{t-1} - \deg{v}\firing{v}{t})
  \leq \deg{v}-1.
\end{equation}
\end{lem}

\begin{proof}
We express $\chips{v}{b}$ in terms of $\chips{v}{a-1}$:
\begin{align*}
  \chips{v}{b} &= \chips{v}{a-1} +
  \smashoperator{\sum_{t=a-1}}^{b-1}(\receiving{v}{t} - \deg{v}\firing{v}{t})
  \\
  \chips{v}{b} - \deg{v}\firing{v}{b} &= \chips{v}{a-1} -
  \deg{v}\firing{v}{a-1} + \sum_{t=a}^{b}(\receiving{v}{t-1} -
  \deg{v}\firing{v}{t}).
\end{align*}
As mentioned in \cite[Section 2]{kominers}, the set of vertices $v$ such that
$\chips{v}{t} \geq 2\deg{v}$ chips is nonincreasing as $t$ increases. In
particular, in a periodic game, this set is empty unless the game has period 1,
in which case the lemma is clear. Otherwise, simple analysis of the waiting and
firing cases shows that $\chips{v}{t} \leq 2\deg{v} - 1$ implies $0 \leq
\chips{v}{t} - \deg{v}\firing{v}{t} \leq \deg{v} - 1$, which gives the desired
inequality.
\end{proof}

We define
\begin{align*}
\pfpverts{p} &= {\set{v \in \V}{\pfp{v}=p}} \\
\pfpedges{p}{q} &= {\set{\{v,w\} \in \E}{\pfp{v}=p, \pfp{w} = q}} \\
\misb{S}{p}{q} &= \sum_{i \in S}(p_{i} - q_{i-1})
\end{align*}
for binary strings $p$ and $q$ and sets of indices $S \subseteq \nats$. Note
that $\misb{S}{p}{q}$ is very closely related to the misbehavior of $q$ towards
$p$, missing only sector-dependent signs. We now have the tools to prove our
main result.

\begin{thm}\label{nct}
Clumpy PFPs do not occur in the parallel chip-firing game.
\end{thm}

\begin{proof}
The key is to examine the mischief between all pairs of neighbor vertices in
which at least one neighbor has a clumpy PFP. Roughly speaking, summing an
inequality derived from Lemma~\ref{strongbg} over all vertices with clumpy PFPs
bounds a negative sum of mischiefs below, and summing the inequality given by
the Theorem~\ref{morale} over all edges incident with a vertex with a clumpy
PFP gives an upper bound on the same sum of mischiefs. The lower bound relates
positively with the number of vertices with clumpy PFPs, and the upper bound is
0.

Let $a,b \in \nats$ and $v \in \V$. Grouping the sum in \eqref{local} by $v$'s
neighbors instead of time steps yields
\[
  -\deg{v} + 1 \leq -\smashoperator{\sum_{w \in
      \N{v}}}\misb{[a,b]}{\pfp{v}}{\pfp{w}} \leq \deg{v} - 1.
\]
Reordering the inequality on the left, recalling that $\deg{v} = \size{\N{v}}$,
gives
\[
  1 \leq \smashoperator{\sum_{w \in \N{v}}}(1 - \misb{[a,b]}{\pfp{v}}{\pfp{w}}),
\]
and by considering the right inequality we obtain the same result with the $-$
switched to $+$. Let $p$ be a PFP. By summing the above over $v \in
\pfpverts{p}$ for some choice of signs, we obtain
\[
  \size{\pfpverts{p}} \leq \smashoperator{\sum_{\substack{v \in \pfpverts{p}
        \\ w \in \N{v}}}}(1 + r_v\misb{[a,b]}{p}{\pfp{w}}),
\]
where each $r_v = \pm1$ may be chosen dependent on $v$. (Notation: domains for
outer sums are above domains for inner sums.) That each
$\misb{[a,b]}{p}{\pfp{w}}$ term is preceded by a sign of our choice is a hint
that we can relate this quantity to misbehavior.

For all PFPs $p$, let $\sctr{p}$ be the set of sectors of $p$. Because each
sector of a PFP is of the form $[a,b]$ for some $a,b \in \nats$, we can sum the
above inequality over $X \in \sctr{p}$ and $p \in \cpfps$ to get
\begin{equation}\label{almost}
  \smashoperator[r]{\sum_{\substack{
        p \in \cpfps \\
        X \in \sctr{p}
  }}}\size{\pfpverts{p}} \leq
  \smashoperator{\sum_{\substack{
        p \in \cpfps \\ v \in \pfpverts{p} \\
        w \in \N{v} \\ X \in \sctr{p}
  }}}(1 + r_{v,X}\misb{X}{p}{\pfp{w}}),
\end{equation}
where each $r_{v,X} = \pm1$ may be chosen dependent on $v$ and $X$.

Let $p$ be a clumpy PFP. If $q$ is a clumpy PFP, then
\begin{equation}\label{misbSub1}
  M(p,q) =
  \smashoperator{\sum_{X \in \sctr{p}}}(s_X(p)\misb{X}{p}{q} - 1) +
  \smashoperator{\sum_{X \in \sctr{q}}}(s_X(q)\misb{X}{q}{p} - 1).
\end{equation}
(Abusing notation slightly, we write $s_X(p)$ instead of $s_i(p)$ if $i \in X
\in \sctr{p}$.) The $-1$ in each sum accounts for the
$-\delta_i(p)-\delta_i(q)$ term in $M_i(p,q)$. If instead $q$ is not clumpy,
then $\sctr{q} = \{[0,\period-1]\}$, so $q$ has no sector switches and
\begin{equation}\label{misbSub2}
  M(p,q) =
  \smashoperator{\sum_{X \in \sctr{p}}}(s_X(p)\misb{X}{p}{q} - 1) +
  s_{[0,\period-1]}(q)\misb{[0,\period-1]}{q}{p}.
\end{equation}
However, $\misb{[0,\period-1]}{q}{p} = 0$ because $p$ and $q$ have the same
length and number of 1s. This means that the mischief between $p$ and $q$ only
depends on the misbehavior of $q$ towards $p$ and sector switches in $p$. This
is important because Lemma~\ref{strongbg} bounds total misbehavior towards a
vertex from all neighbors.

Let $W = \set{v \in \V}{\pfp{v} \in \cpfps}$ be the set of vertices with clumpy
PFPs. Choosing $r_{v,X} = -s_X(p)$ in \eqref{almost} yields
\begin{align*}
  \smashoperator[r]{\sum_{\substack{
        p \in \cpfps \\
        X \in \sctr{p}
      }}}\size{\pfpverts{p}}
  &\leq
  \smashoperator{\sum_{\substack{
        p \in \cpfps \\ v \in \pfpverts{p} \\
        w \in \N{v} \\ X \in \sctr{p}
      }}}(1-s_X(p)\misb{X}{p}{\pfp{w}}) \\
  &=
  \smashoperator{\sum_{\substack{
        p \in \cpfps \\ v \in \pfpverts{p} \\
        w \in \N{v} \cap W \\ X \in \sctr{p}
      }}}(1-s_X(p)\misb{X}{p}{\pfp{w}}) +
  \smashoperator{\sum_{\substack{
        p \in \cpfps \\ v \in \pfpverts{p} \\
        w \in \N{v} \setminus W \\ X \in \sctr{p}
      }}}(1-s_X(p)\misb{X}{p}{\pfp{w}}) \\
  &=
  \smashoperator[l]{\sum_{\substack{
        p,q \in \cpfps \\
        e \in \pfpedges{p}{q}
      }}}\!\!
  \Bigg(\smashoperator[r]{\sum_{X \in \sctr{p}}}(1-s_X(p)\misb{X}{p}{q})
  + \smashoperator{\sum_{X \in \sctr{q}}}(1-s_X(p)\misb{X}{q}{p})\Bigg)
  \\ &\qquad +
  \smashoperator{\sum_{\substack{
        p \in \cpfps \\ v \in \pfpverts{p} \\
        w \in \N{v} \setminus W \\ X \in \sctr{p}
      }}}(1-s_X(p)\misb{X}{p}{\pfp{w}}). \\
\end{align*}
Note that we consider $p$ and $q$ in the sum over $p,q \in \cpfps$ to be
unordered. (One alternative notation is a sum over $\{p,q\} \subseteq \cpfps$.)
We now substitute using \eqref{misbSub1} and \eqref{misbSub2} and apply
Theorem~\ref{morale} to get
\[
  \smashoperator[r]{\sum_{\substack{
        p \in \cpfps \\
        X \in \sctr{p}
      }}}\size{\pfpverts{p}}
  \leq
  -\smashoperator{\sum_{\substack{
        p,q \in \cpfps \\
        e \in \pfpedges{p}{q}
      }}}M(p,q) -
  \smashoperator{\sum_{\substack{
        p \in \cpfps \\
        v \in \pfpverts{p} \\ w \in \N{v} \setminus W
      }}}M(p,F(w))
  \leq 0.
\]
Sets have nonnegative sizes, so $\size{\pfpverts{p}} = 0$ for all $p \in
\cpfps$.
\end{proof}
