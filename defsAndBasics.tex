\section{Preliminaries} \label{pres}

\subsection*{Definitions}
A \emph{parallel chip-firing game} $\s$ on a graph $G = (\V, \E)$ is a sequence
$(\pos{t})_{t \in \nats}$ of ordered tuples with natural number elements
indexed by $\V$. Each tuple represents the chip configuration at a particular
turn, where each element of the tuple is the number of chips on the
corresponding vertex. We define the following for all $v \in \V$:
\begin{align*}
  \N{v} &= \set{w \in \V}{\{v,w\} \in \E} \\
  \deg{v} &= \size{\N{v}} \\
  \chips{v}{t} &= \text{number of chips on $v$ in position $\pos{t}$} \\
  \firing{v}{t} &= \begin{cases}
    0 &\text{ if } \chips{v}{t} \leq \deg[]{v} - 1 \\
    1 &\text{ if } \chips{v}{t} \geq \deg[]{v}
  \end{cases} \\
  \receiving{v}{t} &= \sum_{\mathclap{w \in \N{v}}} \firing{w}{t}.
\end{align*}
In a parallel chip-firing game, $\pos{t}$ induces $\pos{t+1}$. For all $v \in
\V$,
\begin{equation}\label{gameDef}
  \chips{v}{t+1} = \chips{v}{t} + \receiving{v}{t} - \firing{v}{t}\deg[]{v},
\end{equation}
so an initial position suffices to define a game on a given graph. When
$\firing{v}{t} = 0$, we say $v$ \emph{waits} at $t$, and when $\firing{v}{t} =
1$, we say $v$ \emph{fires} at $t$.

A position $\pos{t}$ is \emph{periodic} if and only if there exists $p \in \nats$ such
that $\pos{t} = \pos{t+p}$. The minimum such $p$ for which this occurs is the
\emph{period} of $\s$ and is denoted $\period$. Abusing notation slightly, ``a
period'' of a game $\s$ may also refer to a set of times $\{t, t+1, \dots,
t+\period-1\}$, where $\pos{t}$ is periodic. The parallel chip-firing game is
deterministic and there are finitely many possible positions on a given graph
with a given number of chips, so for any game $\s$, there exists $t_0 \in
\nats$ such that $\pos{t}$ is periodic for all $t \geq t_0$. If the initial
position of a game is periodic, we may also call the game itself periodic.

\subsection*{Notation}
\newlength{\tablespace}
\setlength{\tablespace}{.3\baselineskip}
Definitions for invented notation are given in the section indicated in the
last column.\\

\showgame
\begin{centering}
  \begin{tabular}{l p{.65\textwidth} l}
    \toprule
    \multicolumn{2}{l}{\emph{Parallel Chip-Firing}} & \emph{Defined in} \\
    \midrule

    $\chips{v}{t}$ & Number of chips on vertex $v$ in position $\pos{t}$. &
    Section~\ref{pres} \vspace{\tablespace}\\

    $\pfp{v}$ & Periodic firing pattern of $v$. & Section~\ref{nonclumpiness}
    \vspace{\tablespace}\\

    $\firing{v}{t}$ & Indicates whether or not vertex $v$ fires in $\pos{t}$. &
    Section~\ref{pres} \vspace{\tablespace}\\

    $\receiving{v}{t}$ & Number of chips vertex $v$ will receive in $\pos{t}$.
    & Section~\ref{pres} \vspace{\tablespace}\\

    $\period$ & Period of $\s$. & Section~\ref{pres} \vspace{\tablespace}\\

    $\mots$ & Set of vertices that are motors in $\s$. & Section~\ref{motors}
    \vspace{\tablespace}\\

    $\pfps$ & Set of periodic firing patterns in $\s$. &
    Section~\ref{nonclumpiness} \vspace{\tablespace}\\

    $\cpfps$ & Set of ``clumpy'' periodic firing patterns in $\s$. &
    Section~\ref{nonclumpiness} \vspace{\tablespace}\\

    \toprule
    \multicolumn{3}{l}{\emph{Graphs}} \\
    \midrule

    $\V$ & \multicolumn{2}{l}{Vertex set of graph $G$} \vspace{\tablespace}\\

    $\E$ & \multicolumn{2}{l}{Edge set of graph $G$} \vspace{\tablespace}\\

    $\N[G]{v}$ & \multicolumn{2}{l}{Neighbors of vertex $v$.}
    \vspace{\tablespace}\\

    $\deg[G]{v}$ & \multicolumn{2}{p{.815\textwidth}}{The degree of vertex $v$
      in graph $G$.} \vspace{\tablespace}\\

    \toprule
    \multicolumn{3}{l}{\emph{Other}} \\
    \midrule

    $[a,b]$ & \multicolumn{2}{l}{The integer interval $\{a, a+1, \dots, b\}$.}
    \vspace{2\tablespace}
  \end{tabular}
\end{centering}
\hidegame

We leave out the subscript $G$ or superscript $\s$ if there is no ambiguity.
