\section{Simulating Motors} \label{simulatingMotors}
\showgame

In this section, to refer to multiple chip-firing games unambiguously, we
include the subscripts and superscripts in, for example, $\deg[G]{v}$ and
$\firing{v}{t}$.

We call a firing sequence $(f_t)_{t \in \nats}$ \emph{possible} if there exists
an ordinary game $\s$ on some graph $G$ such that $\firing{v}{t} = f_{t}$ for
all $t \in \nats$. Our next theorem states that we can simulate motorized games
with ordinary games as long as every motor's firing sequence is
possible. Figure~\ref{natMot} demonstrates the concept.

\begin{centering}
\begin{figure}[tbh]
  \subfloat{\includegraphics[width=\figWidthA]{Figures/natMotor}}
  \subfloat{\includegraphics[width=\figWidthB]{Figures/keyShortish2}}
  \caption{Suppose the motor in motorized game (b) has firing sequence
    $(0,0,0,0,1,0,0,0,\dots)$. This occurs in ordinary game (a). By using
    sufficiently many copies of (a) and carefully choosing $n$, we construct
    (c). The behavior of $G$ in (c) is identical to the behavior of $G$ in
    (b).}
\label{natMot}
\end{figure}
\end{centering}

\begin{thm} \label{natMotors}
Let $\s$ be a periodic motorized game on $G$. If every motor's firing sequence
is possible, then there exists an ordinary game $\s'$ on a graph $H \supseteq
G$ such that
\begin{itemize}
\item $\firing[\s']{u}{t} = \firing{u}{t}$ for all $t \in \nats$ and $u \in
  \V$,
\item $\deg[H]{v} = \deg[G]{v}$ for all $v \in \V \setminus \mots$, and
\item the subgraph of $H$ induced by $V(G)$ is $G$. (That is, $H$ contains no
  edges between vertices of $G$ that are not also in $G$.)
\end{itemize}
\end{thm}

\begin{proof}
Our approach will be, for each $m \in \mots$, to attach many copies of a graph
with a vertex with $m$'s firing sequence to $m$. If sufficiently many copies
are attached, the number of chips $m$ has due to its neighbors in $G$ becomes
irrelevant as to whether or not it fires.

For each $m \in \mots$, let $A_m$ be a graph such that there exists a game
$\s^m$ and some vertex $u _m\in \V[A_m]$ such that $\firing[\s^m]{u_m}{t} =
\firing{m}{t}$ for all $t \in \nats$. Let $a_m$ and $b_m$ be the minimum and
maximum respectively of $\set{\chips{m}{t}}{t \in \nats}$. Let $k_m = b_m - a_m
+ 1$. Let $H$ be the union of $G$ and $k_m$ copies of each $A_m$, with $G$ and
the copies of $A_m$ disjoint except for $m = u_m$ for each $m \in
\mots$. (Thinking of the graphs as pointed topological spaces, each with
basepoint $m$ or $u_m$, $H$ is a wedge sum.)

It is clear by construction that $H$ contains no new edges between vertices of
$G$ and that
\begin{itemize}
\item $\deg[H]{m} = k_m\deg[{A_m}]{m} + \deg[G]{m}$ for all $m \in \mots$,
\item $\deg[H]{u} = \deg[{A_m}]{u}$ for all $u \in \V[A_m] \setminus \{m\}$ for
  each $m \in \mots$, and
\item $\deg[H]{v} = \deg[G]{v}$ for all $v \in \V \setminus \mots$.
\end{itemize}
Suppose that for some $t \in \nats$, $\pos[\s']{t}$ satisfies the following.
\begin{enumerate} \label{posAtT}
\item $\chips[\s']{m}{t} = k_m\chips[\s^m]{m}{t} + \deg[G]{m} + \chips{m}{t} -
  a_m$ for all $m \in \mots$.
\item $\chips[\s']{u}{t} = \chips[\s^m]{u}{t}$ for all $u \in \V[A_m] \setminus
  \{m\}$ for each $m \in \mots$.
\item $\chips[\s']{v}{t} = \chips{v}{t}$ for all $v \in \V \setminus \mots$.
\end{enumerate}
We will show that $\pos[\s']{t+1}$ satisfies the above as well. We have
$\deg[H]{v} = \deg[G]{v}$ for all $v \in \V \setminus \mots$, so
$\firing[\s']{v}{t} = \firing{v}{t}$ for all $v \in \V \setminus
\mots$. Similarly, $\firing[\s']{u}{t} = \firing[\s^m]{u}{t}$ for all $u \in
\V[A_m] \setminus \{m\}$ for each $m \in \mots$. Finally, for all $m \in
\mots$, if $\firing[\s^m]{m}{t} = 0$, then
\begin{align*}
  \chips[\s']{m}{t} &\leq k_m(\deg[{A_m}]{m)-1} + \deg[G]{m} + \chips{m}{t} -
  a_m \\
  &= k_m\deg[{A_m}]{m} + \deg[G]{m} + (\chips{m}{t} - b_m) - 1 \\
  &\leq \deg[H]{m} - 1,
\end{align*}
and if $\firing[\s^m]{m}{t} = 1$, then
\begin{align*}
  \chips[\s']{m}{t} &\geq k_m\deg[{A_m}]{m} + \deg[G]{m} + (\chips{m}{t} - a_m)
  \\
  &\geq \deg[H]{m},
\end{align*}
so $\firing[\s']{m}{t} = \firing[\s^m]{m}{t} = \firing{m}{t}$.

We know $\firing[\s']{v}{t} = \firing{v}{t}$ for all $v \in \V[H]$, so clearly
$\chips[\s']{v}{t+1} = \chips{v}{t+1}$ for all $v \in \V[G] \setminus \mots$
and $\chips[\s']{u}{t+1} = \chips[\s^m]{u}{t+1}$ for all $u \in \V[A_m]
\setminus \{m\}$ for each $m \in \mots$. Finally, we have
\begin{align*}
  \chips[\s']{m}{t+1} &= k_m\chips[\s^m]{m}{t} + \deg[G]{m} + \chips{m}{t} -
  a_m\ + \receiving[\s']{v}{t} - \firing[\s']{v}{t}\deg[H]{v} \\
  &= k_m\chips[\s^m]{m}{t} + \deg[G]{m} + \chips{m}{t} - a_m + \receiving{v}{t}
  - \firing{v}{t}\deg[G]{v}\ + \\
  &\qquad k_m\receiving[\s^m]{v}{t} - k_m\firing[\s^m]{v}{t}\deg[{A_m}]{v} \\
  &= k_m(\chips[\s^m]{m}{t} + \receiving[\s^m]{v}{t} -
  \firing[\s']{v}{t}\deg[{A_m}]{v)} + \deg[G]{m}\ + \\
  &\qquad (\chips{m}{t} + \receiving[\s]{v}{t} - \firing[\s]{v}{t}\deg[G]{v)} -
  a_m \\
  &= k_m\chips[\s^m]{m}{t+1} + \deg[G]{m} + \chips{m}{t+1} - a_m.
\end{align*}
for all $m \in \mots$.

We can distribute chips in $\pos[\s']{0}$ such that it satisfies (1), (2), and
(3), in which case, by induction, $\pos[\s']{t}$ satisfies (1), (2), and (3)
for all $t \in \nats$, implying $\firing[\s']{v}{t} = \firing{v}{t}$ for all $v
\in V(G)$.
\end{proof}

Again, we may relax the condition that the game be periodic. In this case, the
periodicity ensures that the number of chips on each motor is bounded. This is
easily shown to be equivalent to each motor having the same activity, so that
is a sufficient condition for the theorem. Of course, because all ordinary
games are eventually periodic, any motorized game in which each motor has a
possible firing sequence will eventually be periodic.

In Theorem~\ref{cheapLunch}, motors were primarily a convenient intuition and
terminology; we could have proved a similar theorem within the context of the
ordinary parallel chip-firing game, though its statement would have been
messier. Theorem~\ref{natMotors} demonstrates another way in which the motor
concept is useful, making certain conjectures easy to prove or disprove by
example.

\hidegame
